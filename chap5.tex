\chapter{GEM detector Data Analysis}

Gas Electron Multiplier(GEM) Detectors are gaseous detectors. The key component of the GEM detector is 5um copper-coated polyimide foils. On the foil, there are 70um holes are etched in a regular hexagonal pattern, and the distance between that holes is 140 micrometers. On the top of the GEM foil is the cathode, and under the bottom is the readout board used for collecting the employed electrons. A potential difference applied across the foils generates sharp electric fields in the holes. The electrons created during the ionization process drift toward the foils and are multiplied in the holes. The resulting electron avalanche induces a readout signal on the finely spaced strips. To achieve a higher amplify ability, multiple layers of GEM detector are stacked together, and with those layers of GEM foils, the ionized electron can be amplified multiple times before reaches the readout board. 

The GEM detectors used for PRex/CRex experiment were designed for Jefferson Lab Super BigBite Spectrometer(SBS) with an area of 50cmx60cm. By the time it was built, it was one of the largest GEM detectors that have ever been built. The PRex/CRex provides a good opportunity to test the GEM detector in a real experiment environment before the start of the SBS experiment. In this chapter, we will cover the apparatus and the analysis result of the GEM detectors. 

[GEM detector structure figure]



\section{GEM detector}
\subsection{GEM detector structure}
\subsection{GEM detector in apparatus}
\subsection{GEM readout electronics}
\subsection{Working HV, Gas}

\section{GEM detector data analysis}
\subsection{GEM detector alignment}
\subsection{GEM detector performance}
\subsubsection{GEM detector cluster size}
\subsubsection{GEM detector tracking efficiency analysis}
\subsubsection{GEM detector efficiency over time}
