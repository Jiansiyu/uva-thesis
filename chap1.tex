\chapter{Introduction (10)}

\href{https://docs.google.com/spreadsheets/d/1sSxj0pFhz2N5CeNrXt-o7Ybs_TAdTg6Rkz5J-D-VANc/edit?usp=sharing}{Timeline}

\begin{enumerate}
    \item 5.8  - 5.14  ch 2
    \item 5.15 - 5.21  ch 2
    \item 5.22 - 5.28  ch 3
    \item 5.29 - 6.4   ch 1
    \item 6.5  - 6.11  ch 6
    \item 6.29 oral
\end{enumerate}



\begin{itemize}
    \item history of the nuclear structure
    \item scatter is used for nuclear structure probe, charge radius 
    \item neutron charges 0, poorly measured
    \item historical measurement 
    \item parity violation measurement of the neutron density
\end{itemize} 

In 1961, Robert Hofstadter wins Nobel Price for his pioneering studies of the structure of nucleons with  electron scattering. Accelerated electron beams are widely used to study the inner structure of the nucleons (and neutrons) since then. That experiment can provide different level structure information of nucleons given different incident electron energy. At the range where the energy of electrons is deficient, $\lambda \gg r_p$, here $r_p$ is the proton's radius, $\lambda$ is the electron wavelength, the scattering is equivalent to scattered over point like spin-less object. At a low electron energy range where $\lambda ~r_p$, the scattering is equivalent to scattering over the charged object. When the electron wavelength $\lambda < r_p$, the electron scattering will be able to see the substructures. At a very high energy range where $\lambda \ll r_p$, the electrons are equivalent to scattering over the sea of quarks and gluons of the protons. 

\section{Neutron Density}
\section{current theory of the neutron density}

\section{Rich Physics Behind the PRex Experiment}
\subsection{Neutron density of neutron-rich matter}
\subsection{neutron density theory and corrections to Apv}
\subsection{nuetron radius measurementys}
\subsection{neutron stars}
\subsection{Gravity waves and EOS}
\subsection{Atomic Parity Non-Conservation Experiment}
\subsection{...}
\section{Previous Experiment}
\subsection{PRex-I experiment}