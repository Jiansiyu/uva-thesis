\chapter{Introduction (10)}

In 1961, Robert Hofstadter wins Nobel Price for his pioneering studies of the structure of nucleons with electron scattering. Accelerated electron beams have been widely used to study the inner structure of the nucleons (and neutrons) since then. That experiment can provide different level structure information of nucleons given different incident electron energy. At the range where the energy of electrons is deficient, $\lambda \gg r_p$, here $r_p$ is the proton's radius, $\lambda$ is the electron wavelength, the scattering is equivalent to scattered over point like spin-less object. At a low electron energy range where $\lambda ~r_p$, the scattering is equivalent to scattering over the charged object. When the electron wavelength $\lambda < r_p$, the electron scattering will be able to see the substructures. At a very high energy range where $\lambda \ll r_p$, the electrons are equivalent to scattering over the sea of quarks and gluons of the protons. 

Para. nucleus structure

Para. 
\begin{itemize}
    \item Hofstadter initiates work
    \item our current understanding of the nucleus structure
\end{itemize}

Para.

\begin{itemize}
\item history of the nuclear structure
    \item nucleus structure, how to understand it
    \item charge radius measurement [merge with chapter 2?]
    \item scatter is used for nuclear structure probe, charge radius 
\end{itemize}


Para. 

\begin{itemize}
    \item EM force, nuclear force
    \item neutron density and assumptions
\end{itemize}


\begin{itemize}
    \item neutron charges 0, poorly measured
    \item historical measurement 
    \item parity violation measurement of the neutron density
\end{itemize} 

\section{Rich Physics Behind the PRex Experiment}
\subsection{Neutron density of neutron-rich matter}
\subsection{neutron density theory and corrections to APV}
\subsection{neutron radius measurements}
\subsection{Gravity waves, EOS, and neutron stars}
\subsection{Atomic Parity Non-Conservation Experiment}

\section{Previous Experiment}
\subsection{current theory of the neutron density}

\section{Historical measurement[to be added, move to chapter 1]}
\subsection{Pb charged radius measurement and its result }

\section{Impact of the experiment, result based on PRex-II experiment [move to chapter 6 conclusion]}

\subsection{calculate with Coulomb energy differences Phys. Lett 29B 396 (1969)}
\subsection{L. Ray and G.W.Hoffmann Physics Rev C 31. 538 (1985)}
\subsection{ Parity Violating Measurements of the Neutron Density, C.J. Horowitz, etc}
\subsection{PRex-I experiment}