\chapter{Experiment Setup ~20}

\section{Jefferson Lab CEBAF}

The PRex and CREx experiments are conducted at the Continuous Electron Beam Accelerator Facility (CEBAF) housed within Jefferson Lab. Since its inception in the early 1990s, CEBAF has been a global pioneer, forging paths in nuclear physics research.

The heart of CEBAF is its linear accelerators (linacs), specifically the North and South linacs. These linacs serve as the key elements where electrons amass energy, essential for the facility's experimental proceedings. Connecting these two linacs are the Recirculation Arcs, which are magnetic systems designed to curve the electron beam, redirecting it back into the linacs for additional energy enhancement during successive passes. An added function of the Recirculation Arcs is to facilitate the separation of electron beams of varying energy levels, post their multiple circulations through the linacs. This structural alignment and the operational interplay between the components ensure precision and efficacy in executing complex nuclear physics experiments like PRex and CRex.


% Para. 2
% \begin{itemize}
%     \item 12GeV upgrade
%     \item RF separator
%     \item RF cavities, cooled 
%     \item Beam energy and current ability
%     \item accelerated up to 5 times though both linacs, producing a nominal energy of 10.9GeV. 
%     \item 1497MhZ split into three 499MHz.
%     \item north and south linacs each can gen 1.1GeV 
%     \item East/West Arc bend the beam to accelerate again 
%     \item Hall ABC, 10.9GeV 
%     \item Hall D 12GeV
% \end{itemize}

The Jefferson Lab's CEBAF upgrade project, completed in 2014, introduced a number of significant enhancements. The modifications included the addition of five cryomodules to each Linac section, each housing 7-cell cavities capable of handling a higher RF field, courtesy of advanced surface treatments.

Each of these upgraded linacs can now facilitate an energy gain of 1.1 GeV, enabling the electron beam to achieve 2.2 GeV during each circulation. Post traversing the south linac, the electron beam can be partitioned into three distinct 499 MHz beams.

The facility's design allows for the electron beam to be accelerated up to five times through both linacs, producing a nominal energy output of 10.9 GeV. Experimental Halls A, B, and C each can receive a beam carrying energy that is one-fifth of the full 5-pass energy. Additionally, the beam directed to Hall D can be re-accelerated in the north linac, reaching energy levels as high as 12 GeV. This upgrade has elevated the capacity and flexibility of the CEBAF, further enhancing its research potential.

% Para. 3
% \begin{itemize}
%     \item PRex requires high polarized beam 
%     \item high current to be able to get enough statistics to be able to measure 
%     \item PRex experiment choose CEBAF
%     \item high current
%     \item polarize beam ability
% \end{itemize}

The PRex-II and CREx experiments aim to quantify the thickness of the neutron skin by analyzing parity-violated asymmetry. To achieve a reliable measurement with a sufficient confidence interval, a significant volume of statistical data is required. This necessity is met by the capabilities of CEBAF, which can generate a polarized electron beam with a current reaching up to $200 uA$ and a polarization level in the range of $80-90\%$. Such robust performance plays a pivotal role in enabling the execution and success of the PRex and CREx experiments.


\begin{figure}
     \centering
     \begin{subfigure}[b]{0.45\textwidth}
         \centering
         \includegraphics[width=\textwidth]{images/chap3/JLab_cebaf_photo.png}
         \caption{Photo of CEBAF}
         \label{Photo of CEBAF}
     \end{subfigure}
     \hfill
     \begin{subfigure}[b]{0.45\textwidth}
         \centering
         \includegraphics[width=\textwidth]{images/chap3/JLab_cebaf.png}
         \caption{JLAB CEBAF}
         \label{JLAB CEBAF}
     \end{subfigure}
\end{figure}

\section{Beam injection}
% Para. 1 & 2
% \begin{itemize}
%     \item key componet to generate polarized beam
%     \item GaAs photocathode
%     \item three different structure $30-40\%$, $50\%$, $100\%$ which is using now. Engy band plot
%     \item plot/structure of the GaAs photocathode
% \end{itemize}

The polarized electron source at Jefferson Lab comprises a specialized polarized laser system and gallium arsenide (GaAs) photocathodes. This integral apparatus permits the generation of a polarized, or spin-aligned, electron beam. This feature is indispensable for a broad range of experiments conducted at our facility, particularly those probing spin-dependent phenomena.

The GaAs photocathode is irradiated with circularly polarized laser light possessing energy exceeding the bandgap energy. This triggers the emission of electrons that exhibit specific spin polarization, a process attributable to the photoelectric effect.

\begin{figure}[!htbp]
    \centering
    \includegraphics[width=\textwidth]{images/chap3/jlab_polarized_source.png}
    \caption{Schema of the Polarized Laser system in the injector. add REF}
    \label{fig:cebaf_polarize_laser_system}
\end{figure}
\subsection{Polarized Beam Source}

The polarization of the electron beam is quantified by the following equation:

\begin{equation}
p = \frac{N{\uparrow} - N{\downarrow}}{N{\uparrow} + N{\downarrow}}
\end{equation}

Here, $N$ represents the number of electrons in the conduction band of each corresponding spin state. 

The generation of polarized electrons in our source is underpinned by the photoemission process from negative-electron-affinity III-V semiconductor materials, notably Gallium Arsenide (GaAs). The electron spin polarization observed arises from the unique crystal symmetry of GaAs, which governs transitions from the fully occupied valence band to the normally vacant conduction band, akin to transitions from a state with total angular momentum j=3/2 to one with j=1/2. For bulk GaAs as showed in the plot [todo ADD THE energy plot], permits two transitions between the valence band and the conduction band. For every given excitation, three times as many electron spins are anti-parallel to the photon spin than parallel to it. as a result, in theory the maximum polarization can achieve is $50\%$. But in practice, the polarization is around $30-40\%$. 


Addressing this limitation, T. Maruyama and colleagues at the Stanford Linear Accelerator (SLAC) pioneered the use of strained GaAs to increase the polarization rate, \cite{PhysRevLett.66.2376}. This approach achieved an enhanced polarization rate of approximately $70\%$. The technique of strained GaAs involves the epitaxial growth of an InGaAs surface layer on GaAs. The resulting crystal-lattice mismatch between these two compounds introduces an axial strain. This strain effectively breaks the degeneracy of the $P_{3/2}$ valence band energy levels, thereby providing a mechanism to restrict optical excitation from the undesired valence-band spin state. This advancement represents a significant stride in the optimization of electron beam polarization. However, strained GaAs has its own challenges, notably a lower quantum efficiency (QE), which is typically on the order of $0.1\%$. Efforts to augment the QE, for instance, by increasing the thickness of the strained layer, tend to result in a concomitant decrease in polarization, presenting a trade-off that must be carefully managed in practical applications.

The modern photocathodes employ a superlattice structure, composed of numerous thin-layer pairs of lattice-mismatched materials. This design amalgamates the advantages of numerous thin strained layers to simultaneously achieve high polarization and high QE. The photocathode uses in Jefferson Lab achieves approximately $90\%$ polarization while maintaining a quantum efficiency of about $1\%$. \cite{PhysRevB.13.5484}, \cite{ADDERLEY2023167710}


[bulk GaAs figure]

[strain GaAs figure]

[super-lattice GaAs figure]

[add energy level figure]

\subsection{Helicity Control}
Helicity plays a pivotal role in the PRex/CRex experiments. For the minimization of systematic errors, it is imperative to quickly and consistently flip the helicity, or spin direction, of the electron beam. The Pockels Cell (PC) is instrumental in facilitating this helicity reversal. This device is indeed crucial for the PRex experiment as it enables rapid, precise control over the electron beam's helicity, a vital requirement for measuring neutron distribution.

The heart of the Pockels cell consists of two piezoelectric crystals made of Rubidium Titanyl Phosphate. When voltage is applied to these crystals, they can transform linearly polarized laser light into either left or right circularly polarized light. This polarization alteration directly affects the helicity of the electron beam. By modulating the voltage applied to the crystals, the beam's helicity can be swiftly adjusted. For the PRex-II experiment, the helicity flip rate was set at 240Hz.[REF pockels cell board paper] 

\section{Beam monitor}

Numerous beam monitors operate within CEBAF (Continuous Electron Beam Accelerator Facility) to ensure its functionality aligns with the experimental requirements. Among these monitors, the polarimeter, beam position monitor, and beam current monitor play significant roles in the success of the PRex/CRex experiments. This section provides a concise introduction to these vital monitoring systems, offering an insight into their operations and their importance in maintaining the integrity of the experiment.

\subsection{polarimeters}

\subsubsection{Mott Polarimeter}



\subsubsection{Compton Polarimeter}

The Compton polarimeter in Jefferson Lab's Continuous Electron Beam Accelerator Facility (CEBAF) Hall A is another essential instrument used to measure the polarization of the electron beam. Unlike the Møller polarimeter, which relies on electron-electron interactions, the Compton polarimeter operates based on the principle of Compton scattering.

Compton scattering involves the interaction between a beam of electrons and a photon, typically originating from a laser. The electrons scatter off the photons and experience a change in energy and direction depending on the incident photon's energy and the scattering angle. The change in energy of the scattered electrons (or alternatively, the scattered photons) is directly related to the polarization of the incident electron beam.


In the Compton polarimeter in Hall A, the beam is directed into the optical cavity by two dipole magnets. A high-powered, circularly polarized laser is strategically positioned inside the cavity to intersect the electron beam. This encounter leads to Compton scattering, thereby generating scattered photons and electrons. However, only a minute fraction of the electron beam interacts with the polarized photons. Post-scattering, the beam is deflected back to the beam line. The scattered electrons are detected using electron detectors, which comprise plates of synthetic diamonds developed through chemical vapor deposition. Compton-scattered photons are identified within an array of lead tungstate (PbWO4) crystals.

A unique advantage of the Compton polarimeter is its capacity to assess the beam's polarization without disrupting its operation. This feature facilitates uninterrupted, real-time monitoring of the beam's polarization during experiments. The disparity in count rates between the two helicity states serves as an indicator of the beam's polarization, thereby enhancing the accuracy and reliability of the measurements.


\begin{figure}[!htbp]
    \centering
    \includegraphics[width=\textwidth]{images/chap3/compton.png}
    \caption{Caption}
    \label{fig:enter-label}
\end{figure}


% \begin{equation}
%     A_{exp} = \frac{n^+-n^-}{n^++n^-} = P_{\gamma}P_e<A_{th}>
% \end{equation}

\subsubsection{Moller Polarimeter}

The Møller polarimeter operates by detecting the electrons scattered from a polarized target, leveraging the phenomenon known as Møller scattering. This type of electron-electron scattering happens when two electrons interact, and the outcome of this interaction is measured. Significantly, the probability of scattering is deeply influenced by the relative spin orientations of the interacting electrons. This property makes Møller scattering particularly beneficial for gauging beam polarization. Unlike the Compton polarimeter, however, the Møller polarimeter cannot operate simultaneously with the production run.

In the ultra-relativistic limit, the Møller asymmetry can be expressed as:

\begin{equation}
A_{exp} = P_{beam}P_{targ}<A_{zz}>
\end{equation}

In this equation, $<A_{zz}>$ represents the average analysis power over the captured cross-section, given by:

\begin{equation}
A_{zz} = \frac{\sin^2{\theta(7 + \cos^2{\theta})}}{(3 + \cos^2{\theta})^2}
\end{equation}

\begin{figure}[!htbp]
    \centering
    \includegraphics[width=\textwidth]{images/chap3/moller.png}
    \caption{moller, need to re-plot}
    \label{fig:enter-label}
\end{figure}

\subsection{Beam Position Monitor}

The beam position is a critical parameter for calculating the kinematics of electron scattering. Various Beam Position Monitors (BPMs) are installed throughout CEBAF to ensure the beam operates optimally. In the PRex-II/CRex experiment, two types of BPMs are mainly employed: one suited for high current beam position measurements, and a low current cavity monitor used when the beam current is low.

Each BPM features a four-wire antenna, placed diagonally along the beamline and denoted as $V_+$, $V_-$, $U_+$, and $U_-$. As the beam traverses the BPM, it induces a current in the antenna, the amplitude of which is proportional to the beam's position and intensity. Figure \ref{fig:cebaf_beam_position_monitor} illustrates the BPM's structure.

The position of the beam along the BPM's antenna-rotated coordinate system can be expressed as:

\begin{equation}
x' = c_x\frac{V_+ - V_-}{V_+ + V_-}
\end{equation}
\begin{equation}
y' = c_y\frac{U_+ - U_-}{U_+ + U_-}
\end{equation}

In these equations, $c_x$ and $c_y$ denote calibration constants. The hall coordinate system is rotated from the BPM's wire direction by $45^\circ$. With the rotation matrix, the position measured by the BPM in the hall coordinate system can be represented as:

\begin{equation}
x_{bpm} = x'\cos{\theta} - y'\sin{\theta}
\end{equation}
\begin{equation}
y_{bpm} = x'\sin{\theta} + y'\cos{\theta}
\end{equation}

Here, the rotation angle $\theta = 45^\circ$. Beam Position Monitor A (IPM1H03A, BPMA) is situated $7.524m$ upstream of the target, and Beam Position Monitor B(IPM1H03B, BPMB) is $1.286m$ upstream of the target.



\begin{itemize}
    \item cavity for low current measurement 
    \item harp for bpm collaboration
\end{itemize}

\subsection{Beam current monitor}

In the PRex-II and CRex experiments conducted at Jefferson Lab, the Beam Current Monitor (BCM) is used for the precise measurements of beam current and beam charge. Engineered for stability, minimal noise, and non-intrusive operation, the BCM is strategically situated about 25 meters upstream of the target.

At its core, the BCM features a Parametric Current Transformer (PCT) sensor, commonly known as an Unser monitor, and a pair of RF cavity monitors. The PCT toroid is designed to be responsive to the direct current (DC) component of the magnetic field, generated by the beam current as it encircles the beam pipe.

To mitigate the impact of parasitic currents, a ceramic gap interrupts the conductivity of the beam pipe within the toroid. Moreover, the system employs three magnetic shields—two comprised of iron and the innermost one of µ metal—to counteract offset drifts induced by fluctuating external magnetic fields.

The entire assembly is encased in a thermoregulated enclosure that further curtails PCT offset drifts. An electrical shield is implemented to prevent high frequencies in the beam spectrum from escaping the monitor via the ceramic gap. Additionally, materials with high absorption capacity are used to shield the sensor from RF noise, ensuring that the BCM consistently delivers accurate and reliable measurements.

\begin{figure}[!htbp]
    \centering
    \includegraphics[width=\textwidth]{images/chap3/beam_current_monitor.png}
    \caption{Caption (SOURCE)}
    \label{fig:enter-label}
\end{figure}


\subsection{Raster}

Generally speaking, the beam size in Jefferson Lab's CEBAF is typically on the order of hundreds of micrometers (${\mu}m$) in diameter. This is a quite small size, which is why the beam can deposit a large amount of energy in a small area and why rastering is necessary when delicate or thin targets are used in experiments to avoid damage to the target.

In Hall A, the raster system includes two sets of air-core dipoles positioned upstream of the Compton polarimeter. Each set of dipoles is driven by an independent power supply, which allows for independent control of the horizontal and vertical dimensions of the rastered beam spot. In the PRex experiment, the rastered beam sport are set to $2cm \times 2 cm$ or $4cm \times 4cm$.

[....... to be added]

Para. to be added
\begin{itemize}
    \item spot++ [used to get the beam plot]
    \item rostered beam plot [find the slides]
    \item calibration of the raster on the beam position which will affect the momentum calibration HRS calibration [slides]
\end{itemize}

\subsection{Beam energy monitor, Tiefenhach measurement}

[..... to be added]

\section{Target System}

The experiments conducted in Jefferson Lab employ two separate target ladders, each hosting different types of targets for specific purposes. The production ladder is equipped with the main targets used for the experiments, while the optics ladder carries calibration targets.

Each target ladder is independently affixed to a motion system and can be controlled separately from the counting-house. As required, these targets can be positioned accurately along the central beamline during the experiment.

\begin{figure}[!htbp]
    \centering
    \includegraphics[width=\textwidth]{images/chap3/target_chamber.jpg}
    \caption{Target Chamber used in PRex-II/CRex experiment}
    \label{fig:target_chamber_photo}
\end{figure}

The production target ladder comprises ten 208Pb-D targets, a Carbon target at $1\%$ density, a Carbon hole target designed for alignment purposes, and a Ca48 target. To maintain their performance under high-intensity beam conditions, the targets on the production ladder are cooled with liquid helium at 15K. Of particular note for the PRex experiment are the 208Pb targets. These targets are approximately 0.5mm thick and sandwiched with diamond foils to enhance thermal conductivity, thereby helping to dissipate the power deposited by the beam.

The optics ladder, on the other hand, houses five targets used for High-Resolution Spectrometer (HRS) optics calibrations. These targets, cooled by water, include a natural Pb target, tungsten target, carbon foil target, carbon hole target, and a falling water target developed by INFN. Among them, the Carbon foil is employed for HRS optics calibration and the water target is for pointing measurements. These measurements enable the accurate determination of the HRS angles. The detailed procedures related to these calibration processes will be thoroughly discussed in Chapter 5.

[... more]

\begin{itemize}
    \item Ca target
    \item carbon hole 
\end{itemize}


\section{High Resolution Spectrometer}

The High-Resolution Spectrometers (HRSs) in Hall A of Jefferson Lab are integral components designed specifically for precision measurements of scattering processes. Capable of high-precision particle detection, these instruments allow scientists to delve into the exploration of subatomic matter structures.

Hall A is equipped with two such HRSs, each capable of measuring momentum with an exceptional resolution of up to $0.01\%$. These spectrometers maintain a broad horizontal angular acceptance of ±28 milliradians (mr) and a vertical acceptance of ±60 mr. Strategically positioned on either side of the beamline, each spectrometer can rotate within a plane perpendicular to the beamline, accommodating angles between 12.5 to 150 degrees. This extensive angular range permits the spectrometers to capture scattered particles across a diverse spectrum of kinematic conditions.

In the context of the PRex and CRex experiments, both spectrometers are adjusted to detect scattered electrons at an angle of 5 degrees to maximizes the number of events collected during the experiment. To complement the minimum acceptance angle of the HRSs, an additional spectrometer magnet is incorporated to pre-bend the particles prior to their entry into the HRS.

Positioned within a shield house, the detector package is composed of a range of detectors, each designed for a specific role in measuring the attributes of scattered particles. The detectors are equipped to assess position, angle, and timing of incoming particles. This trio of information is pivotal in determining their momentum and classification, thus forming a comprehensive overview of the particle's properties.

[.... to be added, the HRS plot]

\subsection{collimeter}
[... merge with the sieve slit collimator]

The collimator, a crucial component in the PREx-II and CREx experiments, is strategically positioned between the target chamber and the septum magnet. It is designed with three holes: the central hole permits the unscattered beam to proceed toward the beam dump, while the two uniquely curved holes on the left and right guide and shape the spectrometer's acceptance.

The design of these spectrometer holes is meticulously optimized to enhance the High-Resolution Spectrometer's (HRS) acceptance for the figure of merit. Another significant part of the collimator is the Sieve slit, as depicted in the plot, which plays a critical role in calibrating the optics.

The collimator features a remote actuation capability, which allows for positioning the Sieve optics collimator between the 'beam out' and 'beam in' positions as required. This design allows for precise adjustments in real time during experiments, providing great control and flexibility in achieving optimal results.

[... add the detailed structure of the collimator]

[figure .... collimator in the beamline]

[figure .... sieve in the beamline]


\begin{figure}
     \centering
     \begin{subfigure}[b]{0.45\textwidth}
         \centering
         \includegraphics[width=\textwidth]{images/chap3/beam_line_target_pivot.png}
         \caption{beam target pivot}
         \label{Photo of CEBAF}
     \end{subfigure}
     \hfill
     \begin{subfigure}[b]{0.45\textwidth}
         \centering
         \includegraphics[width=\textwidth]{images/chap3/hrs_colli_corss_diag.png}
         \caption{disagr[from slides]}
         \label{gem_structure}
     \end{subfigure}
\end{figure}

\begin{figure}[!htbp]
    \centering
    \includegraphics[width=0.8\textwidth]{images/chap3/hrs_pivot_detailed.png}
    \caption{detailed, need to add the labels}
    \label{fig:enter-label}
\end{figure}

\subsection{Sieve slit Collimators}

The sieve slit collimators used in Jefferson Lab's Hall A High Resolution Spectrometers (HRS) are 0.2-inch-thick tungsten plates, precisely drilled with a regular pattern of holes. Each hole measures 0.05 inches in diameter, except for three larger holes that are 0.106 inches across, which are utilized to aid in the accurate location of the collimator holes.
\begin{figure}[!htbp]
    \centering
    \includegraphics[width=0.8\textwidth]{images/chap3/sive_slit.png}
    \caption{Sieve slit used for PRex-II/CRex experiment [need to redraw]}
    \label{fig:enter-label}
\end{figure}
Each side of the HRS beamline is equipped with one sieve slit positioned at the beamline's entrance. The manufacture of these sieve slits involves high-precision Computer Numerical Control (CNC) machining and a thorough surveying process using laser technology, ensuring the utmost accuracy before the commencement of any experiment. The momentum of electrons passing through each sieve hole can be computed precisely, providing a robust reference for spectrometer calibration. Further details regarding the HRS calibration with sieve slits will be discussed in the subsequent chapter.

When the sieve slits are in place, only the electrons passing through these holes can reach the HRS. Each sieve slit is linked to a control unit, enabling manual placement or removal of the slit. During the production runs, the sieve is typically removed and only reinserted during optics runs. To ascertain the accuracy of this system, multiple optics runs were carried out throughout the experiment. 

\subsection{Magnet Chain of the HRS}

[ intro to be added]

\subsubsection{Septum Magnet}

The High-Resolution Spectrometers (HRS) in Jefferson Lab's Hall A possess a size constraint which establishes a minimum measurement angle of $12.5^\circ$. However, the requirements of the PREx-II and CREx experiments necessitate measurements to be conducted at an angle of $5^\circ$. To reconcile this discrepancy, a septum magnet is strategically placed between the collimator and the Q1 quadrupole magnet. This arrangement allows the septum magnet to pre-bend the scattered beam, thus aligning with the minimum angle constraint of the HRS.

\begin{figure}
    \centering
    \includegraphics[width=0.8\textwidth]{images/chap3/septum_colli_targ_photo.jpg}
    \caption{Caption}
    \label{fig:enter-label}
\end{figure}

The septum magnet utilized in these experiments is a conventional, non-superconducting magnet. As depicted in the figure (to be added), the magnet is composed of coils positioned on the left and right beamlines. The magnetic field created by these coils is oriented vertically, causing the scattered electrons to be deflected towards the HRS while the unscattered beam continues along the central beamline towards the beam dump.

The Septum Magnet utilizes a 3-coil configuration, which offers two significant benefits. Firstly, it minimizes the vertical bore dimension, which is vital due to the confined spatial conditions. Secondly, this configuration reduces the coil current, which in turn decreases the thermal power relative to the required field integral. 


\subsubsection{QQDQ magnet package}

[from gpt, whether need to add?]

[.... plot, HRS QQDQ layout plot]

The High-Resolution Spectrometer (HRS) in Jefferson Lab's Hall A employs a quadrupole-quadrupole-dipole-quadrupole (QQDQ) magnetic configuration as the primary component of its particle tracking system.

The QQDQ magnetic package is designed to provide high-resolution momentum analysis and precise particle tracking, crucial for the HRS's function. It consists of two quadrupole magnets (Q1 and Q3), one dipole magnet (D), and another quadrupole magnet (Q2), arranged sequentially in the stated order.

The role of the quadrupole magnets is to focus the particle beam onto the plane of the detector. Specifically, the first and third quadrupole magnets (Q1 and Q3) act as vertical focusing lenses, while the second quadrupole magnet (Q2) serves as a horizontal focusing lens.

The dipole magnet (D) lies between the Q2 and Q3 quadrupoles. This magnet has a significant role in the QQDQ configuration as it deflects the particle beam through a particular angle, thereby separating the particles based on their momentum. This momentum dispersion feature of the dipole magnet allows the HRS to discern and analyze particles of differing momenta.

The unique combination of the quadrupoles and dipole in the QQDQ configuration grants the HRS excellent momentum resolution, precise trajectory determination, and a large angular acceptance. As such, this makes the Jefferson Lab's HRS an incredibly versatile and precise instrument for a wide range of nuclear physics experiments.

\begin{figure}
    \centering
    \includegraphics[width=0.8\textwidth]{images/chap5/gem_apparatus_in_hrs_2d.png}
    \caption{HRS QQDQ chain, to be replaced}
    \label{fig:enter-label}
\end{figure}

\subsection{detector package}

This detector package is an integral part of the spectrometer's functionality, designed for efficient particle detection, identification, and momentum analysis. The HRS detector package includes a collection of detector devices, each offering distinct functionalities, which together enable comprehensive particle detection and analysis. The PRex-II/CRex experiment detector package includes S0, S3 scintillators which provide triggers for counting mode Data Acquisition, Vertical Drift Chambers for particle tracking, Quartz integrating detectors, and GEM detectors which provide supplemental tracking capabilities.  


\subsection{Vertical Drift Chamber}

The Vertical Drift Chamber(VDC) are key components in determining the trajectory of particles detected in the spectrometer. The VDCs are multi-wire proportional chambers that provide precise measurements of the positions and angles of charged particles. Each VDC in the HRS typically consists of two wire planes, with the wires in each plane oriented at an angle with respect to each other. This configuration allows for the determination of the two-dimensional position of a charged particle as it passes through the chamber. With two chambers, the VDCs can provide two dimension position x,y together with the oritation of the particle $\theta$, $\phi$. With those information, the kinematic measurement of the scattered electrons are made possible. In the Chapter 4, we will introduce more details about the VDC, and how to reconstructe the kenematic information with the VDC measurement. 
\begin{itemize}
    \item structure of the VDC 
    \item readout electronics
\end{itemize}

\subsection{S0, S3 scintillator as trigger}
\subsection{Quartz integrating detectors}
\subsection{GEM detector}

\section{GEM detector for the PRex/CRex experiment}

Gas Electron Multiplier (GEM) detectors are an important type of detector used in high-energy physics and various other fields for detecting charged particles, X-rays, and other ionizing radiation. The GEM detector was invented by Fabio Sauli at CERN in 1997 and has since become a crucial tool in the field of particle physics.

The GEM detector operates based on the principle of gas ionization and subsequent avalanche multiplication. It is essentially a thin, metal-clad polymer foil chemically pierced by a high density of holes. This foil is typically sandwiched between two electrode plates. When a voltage is applied across the electrodes, an electric field is generated in the holes, resulting in strong gas amplification of the ionization electrons produced in the detector gas by the traversing radiation.

When a charged particle passes through the detector, it ionizes the gas atoms, generating primary ion-electron pairs. These electrons are then drifted towards the GEM foils due to the electric field. When they enter the high-field region within the holes of the GEM foil, they trigger an avalanche multiplication, leading to a significant increase in the number of electrons.

\begin{figure}
     \centering
     \begin{subfigure}[b]{0.45\textwidth}
         \centering
         \includegraphics[width=\textwidth]{images/chap3/gem_foil_photo.png}
         \caption{Photo of GEM}
         \label{Photo of CEBAF}
     \end{subfigure}
     \hfill
     \begin{subfigure}[b]{0.45\textwidth}
         \centering
         \includegraphics[width=\textwidth]{images/chap3/gem_foil_micro.png}
         \caption{JLAB GEM to be replaced}
         \label{gem_structure}
     \end{subfigure}
\end{figure}

[ plot ..... GEM detector Garfield Simulation EM field plot, my master thesis]

This multiplication process occurs in a confined geometry, reducing the likelihood of photon-mediated feedback and increasing the detector's gain, rate capability, and stability. The resulting electron avalanche then induces a signal on the readout electrode, which can be processed to give the position and amount of ionizing radiation.

GEM detectors are known for their high spatial resolution, good time resolution, and capability of handling high particle rates, making them an ideal choice for many experiments. Furthermore, they are robust, non-ageing, and relatively easy to construct and operate, making them a practical choice for many applications.

In the PRex-II experiment, each HRS have 6 GEM detectors, three of them are UVa GEM detector made for Super bigBite Experiment with active area of $60\times 50$ centimeter, the rest three GEM detector are made in Idohle University with area of $20 \times 10$ centimeters. All the 6 GEM detector can detector particle separately and with the 6 GEM detectors it could pride high accurate tracking of the particles in the spectrometer.  

[ plot, GEM detector installation in the detector hub]
\begin{figure}
    \centering
    \includegraphics[width=0.8\textwidth]{images/chap3/gem_in_hrs.png}
    \caption{Caption}
    \label{fig:enter-label}
\end{figure}

\subsection{GEM detector Structure}

The GEM detectors are constructed with a layered design to ensure sufficient electron amplification. As illustrated in Figure [ref], Below the three layers of GEM foil, readout strips are secured to a honeycomb board. The use of a honeycomb structure minimizes the material present in the beamline, thus reducing its influence on scattered electrons and lessening potential noise.

Underneath the honeycomb board, there exists an additional chamber circulating air to prevent the board from warping. Above the three layers of GEM foil, a cathode is installed. This cathode is covered with a single layer of copper-coated polyimide that connects it to the high-voltage power supply.

The cathode layer is covered by a polyimide window. Positioned above it, an aluminum-coated layer is installed. This layer is connected to the same high voltage as the cathode. This design will depolarize the polyimide window. Through this detailed structure, the GEM detectors ensure efficient detection and measurement of charged particles.


\begin{figure}
     \centering
     \begin{subfigure}[b]{0.2\textwidth}
         \centering
         \includegraphics[width=\textwidth]{images/chap3/gem_structure.png}
         \caption{Photo of GEM [to be replaced]}
         \label{Photo of CEBAF}
     \end{subfigure}
     \hfill
     \begin{subfigure}[b]{0.45\textwidth}
         \centering
         \includegraphics[width=\textwidth]{images/chap3/gem_chamber.png}
         \caption{JLAB GEM to be replaced}
         \label{gem_structure}
     \end{subfigure}
\end{figure}

\subsection{GEM detector High Voltage}

The voltages applied to the GEM foils are regulated by resistor chains. Figure [ref] depicts the specific resistors utilized in the PRex/CRex experiment. The depolarizing window at the top of the GEM chamber shares the same high voltage as the cathode layer. The voltage at the top of the GEM detector is slightly higher than that at the bottom because there are fewer avalanche electrons at the top. This voltage distribution helps to protect the GEMs from potential damage.

Each of the GEM detectors is safeguarded by resistors of $10M\Omega$ to limit the current passing through the GEM. This precautionary measure helps to prevent short circuits or the occurrence of a large drift current within the GEM, ensuring the overall safety and performance of the detector.


\begin{figure}
    \centering
    \includegraphics[width=\textwidth]{images/chap3/GEM_register_chain.png}
    \caption{Caption}
    \label{fig:enter-label}
\end{figure}


The high voltage is powered with two CEAN high voltage modules. Each module has 8 channels, 6 of which are used to power the GEM detectors, each GEM detector is connected to one high voltage channel.  Both of the modules can be controlled remotely in the counting-house. 

\begin{figure}[!htbp]
    \centering
    \includegraphics[width=0.45\textwidth]{images/chap3/gem_hv_connector.png}
    \caption{Caption}
    \label{fig:enter-label}
\end{figure}

[GEM .... GEM detector HV module]


The operational voltage of the GEM detector is selected as the lowest possible voltage that still fulfills the experimental efficiency requirements. When the high voltage is elevated, the potential within the GEM holes increases, which leads to a larger number of avalanche electrons and subsequently higher detection efficiency. However, a higher voltage also elevates the risk of GEM trips, which can potentially damage the GEM detector.

On the contrary, a lower high voltage leads to diminished efficiency. The operational voltage is thus chosen based on the results of a high-voltage scan. This scan evaluates the efficiency of the GEM detector across a range of high voltages to identify the optimal operating condition that balances efficiency and the safety of the device.

[GEM detector efficiency scan]

\subsection{Gas Flow}

In the Gas Electron Multiplier (GEM) detector, the working gas is a mixture of Argon (Ar) and Carbon Dioxide (CO2) in a proportion of 75:25. The primary component, Argon, serves as the ionization medium. Incident radiation interacts with the Argon atoms upon entering the detector, ionizing them and liberating electrons. To prevent the formation of Argon ions after the initial ionization process, which could cause distortions in the electric field, Carbon Dioxide is included in the mixture. As an electronegative gas, CO2 captures and neutralizes these ions, mitigating potential field distortions.

The gas supply is managed via the Hall A gas system. After passing through the gauge valve, the gas is divided into six channels and directed to a bubbler. Post-bubbler, the gas proceeds through a dehumidifier valve before being fed into the GEM chambers. This process ensures an optimal, controlled gas environment within the detectors for precise experimental conditions.

\subsection{GEM detector Readout Electronics}

The GEM detector used in the PRex-II/CRex experiment utilizes a strip readout method. Following the amplification cascade, the electrons are collected by readout strips. Each of these strips is directly linked to a charge-sensitive frontend amplifier. Subsequently, the signals are transformed into LVDS (Low-Voltage Differential Signaling) format and dispatched to a digitizer, where they are converted into digital signals. A schematic diagram of the readout electronics is shown in Figure [ref]. The upcoming sections will offer an in-depth description of each part of the electronics system.

[... to be added, add the readout electronics diagram]

\subsubsection{Front End}

[... to be added, GEM detector with APV]

The GEM detector utilizes APV25, a low-noise charge-sensitive preamplifier for readout operations\cite{Jones:1069892,Jones:432224}. Initially designed for reading silicon microstrip detectors in the CMS tracker at the LHC, APV25 is a 128-channel analogue pipeline chip. Each channel consists of a low-noise amplifier, a 192-cell analogue pipeline, and a deconvolution readout circuit. Data output is transmitted via a single differential current output through an analogue multiplexer.

The dimensions of the GEM detectors are 60 cm by 50 cm. The longer side, equipped with 1536 channels, is read out by twelve APV25 cards. All twelve cards are connected to a single 12-slot backplane that provides power, control, and readout functionality to the APV25. On the other hand, the shorter side, with 1280 channels, is read out by ten APV25 cards. These cards are linked to two 5-slot backplanes. The APV25 is powered by a low voltage regulator that can provide 2.5V and 1.25V to energize the APV25 frontend cards.

Two distinct patch panels are utilized within the data acquisition (DAQ) system. The digital patch panel serves the purpose of distributing digital signals to two different backplanes. On the other hand, the analog patch panel is specifically designed to convert a 5-channel HDMI signal into a 4-channel HDMI signal. This conversion is essential to facilitate the signal's compatibility with the MPD.


[... fig, cooling] \\

[... fif, GEM with APV]


\subsubsection{MPD and Data Transfer}

The Multi-Purpose Digitizer (MPD) board is designed to seamlessly manage up to 16 APV front-end cards. It reads out the associated analog data streams and transmits both the control and configuration signals. Each MPD comes with 4 analog HDMI input ports, each capable of digitizing 4 APV front-end cards (128 channels). The MPD can sample the signal at a rate of 40MHz. After digitization, the signals are forwarded to [??? figure name?] before being transferred to the database. (This needs to be double-checked.)

\section{production DAQ system}

