\chapter{Experiment Setup ~20}

\section{Jefferson Lab CEBAF}

The PRex and CREx experiments are conducted at the Continuous Electron Beam Accelerator Facility (CEBAF) housed within Jefferson Lab. Since its inception in the early 1990s, CEBAF has been a global pioneer, forging paths in nuclear physics research.

The heart of CEBAF is its linear accelerators (linacs), specifically the North and South linacs. These linacs serve as the key elements where electrons amass energy, essential for the facility's experimental proceedings. Connecting these two linacs are the Recirculation Arcs, which are magnetic systems designed to curve the electron beam, redirecting it back into the linacs for additional energy enhancement during successive passes. An added function of the Recirculation Arcs is to facilitate the separation of electron beams of varying energy levels, post their multiple circulations through the linacs. This structural alignment and the operational interplay between the components ensure precision and efficacy in executing complex nuclear physics experiments like PRex and CRex.


% Para. 2
% \begin{itemize}
%     \item 12GeV upgrade
%     \item RF separator
%     \item RF cavities, cooled 
%     \item Beam energy and current ability
%     \item accelerated up to 5 times though both linacs, producing a nominal energy of 10.9GeV. 
%     \item 1497MhZ split into three 499MHz.
%     \item north and south linacs each can gen 1.1GeV 
%     \item East/West Arc bend the beam to accelerate again 
%     \item Hall ABC, 10.9GeV 
%     \item Hall D 12GeV
% \end{itemize}

The Jefferson Lab's CEBAF upgrade project, completed in 2014, introduced a number of significant enhancements. The modifications included the addition of five cryomodules to each Linac section, each housing 7-cell cavities capable of handling a higher RF field, courtesy of advanced surface treatments.

Each of these upgraded linacs can now facilitate an energy gain of 1.1 GeV, enabling the electron beam to achieve 2.2 GeV during each circulation. Post traversing the south linac, the electron beam can be partitioned into three distinct 499 MHz beams.

The facility's design allows for the electron beam to be accelerated up to five times through both linacs, producing a nominal energy output of 10.9 GeV. Experimental Halls A, B, and C each can receive a beam carrying energy that is one-fifth of the full 5-pass energy. Additionally, the beam directed to Hall D can be re-accelerated in the north linac, reaching energy levels as high as 12 GeV. This upgrade has elevated the capacity and flexibility of the CEBAF, further enhancing its research potential.

% Para. 3
% \begin{itemize}
%     \item PRex requires high polarized beam 
%     \item high current to be able to get enough statistics to be able to measure 
%     \item PRex experiment choose CEBAF
%     \item high current
%     \item polarize beam ability
% \end{itemize}

The PRex-II and CREx experiments aim to quantify the thickness of the neutron skin by analyzing parity-violated asymmetry. To achieve a reliable measurement with a sufficient confidence interval, a significant volume of statistical data is required. This necessity is met by the capabilities of CEBAF, which can generate a polarized electron beam with a current reaching up to $200 uA$ and a polarization level in the range of $80-90\%$. Such robust performance plays a pivotal role in enabling the execution and success of the PRex and CREx experiments.


\begin{figure}
     \centering
     \begin{subfigure}[b]{0.45\textwidth}
         \centering
         \includegraphics[width=\textwidth]{images/chap3/JLab_cebaf_photo.png}
         \caption{Photo of CEBAF}
         \label{Photo of CEBAF}
     \end{subfigure}
     \hfill
     \begin{subfigure}[b]{0.45\textwidth}
         \centering
         \includegraphics[width=\textwidth]{images/chap3/JLab_cebaf.png}
         \caption{JLAB CEBAF}
         \label{JLAB CEBAF}
     \end{subfigure}
\end{figure}

\section{Beam injection}
% Para. 1 & 2
% \begin{itemize}
%     \item key componet to generate polarized beam
%     \item GaAs photocathode
%     \item three different structure $30-40\%$, $50\%$, $100\%$ which is using now. Engy band plot
%     \item plot/structure of the GaAs photocathode
% \end{itemize}

The polarized electron source at Jefferson Lab comprises a specialized polarized laser system and gallium arsenide (GaAs) photocathodes. This integral apparatus permits the generation of a polarized, or spin-aligned, electron beam. This feature is indispensable for a broad range of experiments conducted at our facility, particularly those probing spin-dependent phenomena.

The GaAs photocathode is irradiated with circularly polarized laser light possessing energy exceeding the bandgap energy. This triggers the emission of electrons that exhibit specific spin polarization, a process attributable to the photoelectric effect.

\begin{figure}[!htbp]
    \centering
    \includegraphics[width=\textwidth]{images/chap3/jlab_polarized_source.png}
    \caption{Schema of the Polarized Laser system in the injector. add REF}
    \label{fig:cebaf_polarize_laser_system}
\end{figure}
\subsection{Polarized Beam Source}

The polarization of the electron beam is quantified by the following equation:

\begin{equation}
p = \frac{N{\uparrow} - N{\downarrow}}{N{\uparrow} + N{\downarrow}}
\end{equation}

Here, $N$ represents the number of electrons in the conduction band of each corresponding spin state. 

The generation of polarized electrons in our source is underpinned by the photoemission process from negative-electron-affinity III-V semiconductor materials, notably Gallium Arsenide (GaAs). The electron spin polarization observed arises from the unique crystal symmetry of GaAs, which governs transitions from the fully occupied valence band to the normally vacant conduction band, akin to transitions from a state with total angular momentum j=3/2 to one with j=1/2. For bulk GaAs as showed in the plot [todo ADD THE energy plot], permits two transitions between the valence band and the conduction band. For every given excitation, three times as many electron spins are anti-parallel to the photon spin than parallel to it. as a result, in theory the maximum polarization can achieve is $50\%$. But in practice, the polarization is around $30-40\%$. 


Addressing this limitation, T. Maruyama and colleagues at the Stanford Linear Accelerator (SLAC) pioneered the use of strained GaAs to increase the polarization rate, \cite{PhysRevLett.66.2376}. This approach achieved an enhanced polarization rate of approximately $70\%$. The technique of strained GaAs involves the epitaxial growth of an InGaAs surface layer on GaAs. The resulting crystal-lattice mismatch between these two compounds introduces an axial strain. This strain effectively breaks the degeneracy of the $P_{3/2}$ valence band energy levels, thereby providing a mechanism to restrict optical excitation from the undesired valence-band spin state. This advancement represents a significant stride in the optimization of electron beam polarization. However, strained GaAs has its own challenges, notably a lower quantum efficiency (QE), which is typically on the order of $0.1\%$. Efforts to augment the QE, for instance, by increasing the thickness of the strained layer, tend to result in a concomitant decrease in polarization, presenting a trade-off that must be carefully managed in practical applications.

The modern photocathodes employ a superlattice structure, composed of numerous thin-layer pairs of lattice-mismatched materials. This design amalgamates the advantages of numerous thin strained layers to simultaneously achieve high polarization and high QE. The photocathode uses in Jefferson Lab achieves approximately $90\%$ polarization while maintaining a quantum efficiency of about $1\%$. \cite{PhysRevB.13.5484}, \cite{ADDERLEY2023167710}


[bulk GaAs figure]

[strain GaAs figure]

[super-lattice GaAs figure]

[add energy level figure]

\subsection{Helicity Control}
Helicity plays a pivotal role in the PRex/CRex experiments. For the minimization of systematic errors, it is imperative to quickly and consistently flip the helicity, or spin direction, of the electron beam. The Pockels Cell (PC) is instrumental in facilitating this helicity reversal. This device is indeed crucial for the PRex experiment as it enables rapid, precise control over the electron beam's helicity, a vital requirement for measuring neutron distribution.

The heart of the Pockels cell consists of two piezoelectric crystals made of Rubidium Titanyl Phosphate. When voltage is applied to these crystals, they can transform linearly polarized laser light into either left or right circularly polarized light. This polarization alteration directly affects the helicity of the electron beam. By modulating the voltage applied to the crystals, the beam's helicity can be swiftly adjusted. For the PRex-II experiment, the helicity flip rate was set at 240Hz.[REF pockels cell board paper] 

\section{Beam monitor}

Numerous beam monitors operate within CEBAF (Continuous Electron Beam Accelerator Facility) to ensure its functionality aligns with the experimental requirements. Among these monitors, the polarimeter, beam position monitor, and beam current monitor play significant roles in the success of the PRex/CRex experiments. This section provides a concise introduction to these vital monitoring systems, offering an insight into their operations and their importance in maintaining the integrity of the experiment.

\subsection{polarimeters}

\subsubsection{Mott Polarimeter}



\subsubsection{Compton Polarimeter}

The Compton polarimeter in Jefferson Lab's Continuous Electron Beam Accelerator Facility (CEBAF) Hall A is another essential instrument used to measure the polarization of the electron beam. Unlike the Møller polarimeter, which relies on electron-electron interactions, the Compton polarimeter operates based on the principle of Compton scattering.

Compton scattering involves the interaction between a beam of electrons and a photon, typically originating from a laser. The electrons scatter off the photons and experience a change in energy and direction depending on the incident photon's energy and the scattering angle. The change in energy of the scattered electrons (or alternatively, the scattered photons) is directly related to the polarization of the incident electron beam.


In the Compton polarimeter in Hall A, the beam is directed into the optical cavity by two dipole magnets. A high-powered, circularly polarized laser is strategically positioned inside the cavity to intersect the electron beam. This encounter leads to Compton scattering, thereby generating scattered photons and electrons. However, only a minute fraction of the electron beam interacts with the polarized photons. Post-scattering, the beam is deflected back to the beam line. The scattered electrons are detected using electron detectors, which comprise plates of synthetic diamonds developed through chemical vapor deposition. Compton-scattered photons are identified within an array of lead tungstate (PbWO4) crystals.

A unique advantage of the Compton polarimeter is its capacity to assess the beam's polarization without disrupting its operation. This feature facilitates uninterrupted, real-time monitoring of the beam's polarization during experiments. The disparity in count rates between the two helicity states serves as an indicator of the beam's polarization, thereby enhancing the accuracy and reliability of the measurements.


\begin{figure}[!htbp]
    \centering
    \includegraphics[width=\textwidth]{images/chap3/compton.png}
    \caption{Caption}
    \label{fig:enter-label}
\end{figure}


% \begin{equation}
%     A_{exp} = \frac{n^+-n^-}{n^++n^-} = P_{\gamma}P_e<A_{th}>
% \end{equation}

\subsubsection{Moller Polarimeter}

The Møller polarimeter operates by detecting the electrons scattered from a polarized target, leveraging the phenomenon known as Møller scattering. This type of electron-electron scattering happens when two electrons interact, and the outcome of this interaction is measured. Significantly, the probability of scattering is deeply influenced by the relative spin orientations of the interacting electrons. This property makes Møller scattering particularly beneficial for gauging beam polarization. Unlike the Compton polarimeter, however, the Møller polarimeter cannot operate simultaneously with the production run.

In the ultra-relativistic limit, the Møller asymmetry can be expressed as:

\begin{equation}
A_{exp} = P_{beam}P_{targ}<A_{zz}>
\end{equation}

In this equation, $<A_{zz}>$ represents the average analysis power over the captured cross-section, given by:

\begin{equation}
A_{zz} = \frac{\sin^2{\theta(7 + \cos^2{\theta})}}{(3 + \cos^2{\theta})^2}
\end{equation}

\begin{figure}[!htbp]
    \centering
    \includegraphics[width=\textwidth]{images/chap3/moller.png}
    \caption{moller, need to re-plot}
    \label{fig:enter-label}
\end{figure}

\subsection{Beam Position Monitor}

The beam position is an important parameter used for computing the kinematics of electron scattering. There is a couple of beam position monitor in different location of the CEBAF to make sure the beam is operating as expected. In the PRex-II/CRex experiment, two types of beam position monitors are mainly used for the beam position measurement. BPM which used for high current beam position measurement and one low current cavity monitor is used for beam position measurement when the beam current is low. 

\begin{figure}[!htbp]
    \centering
    \includegraphics[width=\textwidth]{images/chap3/beam_position_monitor.png}
    \caption{bpm temp usage, need to replot}
    \label{fig:cebaf_beam_position_monitor}
\end{figure}

Each of the beam position monitors contains a four-wire antenna placed diagnosed along the beamline which denotes as $V_+$, $V_-$, $U_+$, $U_-$.  When the beam passes through the beam position monitor, the beam will induce a current in the antenna, and the amplitude of  the induced current is proportional to the position of the beamline and the intensity of the beam. Figure \ref{fig:cebaf_beam_position_monitor} shows the structure of the beam position monitor. The beam position along the beam position monitor antenna rotated coordination system can be written as: 

\begin{equation}
    x' = c_x\frac{V_+ - V_-}{V_+ + V_-}
\end{equation}


\begin{equation}
    y' = c_y\frac{U_+ - U_-}{U_+ + U_-}    
\end{equation}

In the equation $c_x$,$c_y$ is the calibration constant. The hall coordinate system is rotated the beam position monitor wire direction by $45^\circ$. With the rotation metrics, the position measured with the BPM in the hall coordinate system can be written as: 

\begin{equation}
    x_{bpm} =   x'\cos{\theta} - y'\sin{\theta} 
\end{equation}

\begin{equation}
    y_{bpm} = \x'sin{\theta} + y'\cos{\theta}
\end{equation}

In the equation the rotation angle $\theta = 45^\circ$. Beam position monitor A (IPM1H03A, BPMA) is located $7.524m$ upstream of the target, and Beam position monitor B(IPM1H03B, BPMB) is located $1.286M$ upstream if the target. 



\begin{itemize}
    \item cavity for low current measurement 
    \item harp for bpm collaboration
\end{itemize}

\subsection{Beam current monitor}

In the PRex-II and CRex experiments conducted at Jefferson Lab, the Beam Current Monitor (BCM) is used for the precise measurements of beam current and beam charge. Engineered for stability, minimal noise, and non-intrusive operation, the BCM is strategically situated about 25 meters upstream of the target.

At its core, the BCM features a Parametric Current Transformer (PCT) sensor, commonly known as an Unser monitor, and a pair of RF cavity monitors. The PCT toroid is designed to be responsive to the direct current (DC) component of the magnetic field, generated by the beam current as it encircles the beam pipe.

To mitigate the impact of parasitic currents, a ceramic gap interrupts the conductivity of the beam pipe within the toroid. Moreover, the system employs three magnetic shields—two comprised of iron and the innermost one of µ metal—to counteract offset drifts induced by fluctuating external magnetic fields.

The entire assembly is encased in a thermoregulated enclosure that further curtails PCT offset drifts. An electrical shield is implemented to prevent high frequencies in the beam spectrum from escaping the monitor via the ceramic gap. Additionally, materials with high absorption capacity are used to shield the sensor from RF noise, ensuring that the BCM consistently delivers accurate and reliable measurements.

\begin{figure}[!htbp]
    \centering
    \includegraphics[width=\textwidth]{images/chap3/beam_current_monitor.png}
    \caption{Caption (SOURCE)}
    \label{fig:enter-label}
\end{figure}


\subsection{Raster}

[....... to be added]


\begin{itemize}
    \item spot size 
    \item raster 
    \item spot++
    \item calibration of the raster on the beam position which will affect the momentun calibration HRS calibration
\end{itemize}

\subsection{Beam energy monitor, Tiefenhach measurement}

[..... to be added]

\section{Target System}

\begin{itemize}
    \item target chamber
    \item target ladder 
    \item different target type and its usage
\end{itemize}

\section{High Resolution Spectrometer}

[intro]

\begin{itemize}
    \item core part of the Hall A 
    \item magnet package
    \item septum magnet
    \item tracking detector (VDC, GEM)
    \item AT
    \item trigger
\end{itemize}

\subsection{Sieve plane}

\begin{itemize}
    \item provides the ground truth dataset for the calibration. (Regression model)
    \item CAD structure of the sieve 
    \item location of the sieve. sieve in / out
    \item merge some of the plot from next chamber
    \item  will give a detailed introduction of how to leverage the sieve to calibrate the spectrometer
\end{itemize}

\subsection{Magnet Chain of the HRS}
\subsubsection{Septum Magnet}
\subsubsection{QQDQ magnet package}
\subsection{Vertical Drift Chamber}

\begin{itemize}
    \item GEM foil
    \item GEM foil structure, microscope image
    \item GEM foil magnetic field simulation (Garfield simulation)
\end{itemize}


\begin{itemize}
    \item GEM detector structure
    \item AI window used de-polarized the window
    \item Window 
    \item 3 GEM layers
    \item Read out strips 
    \item backbone 
    \item back chamber used for balance the pressure, prevent the readout board bend
\end{itemize}


\begin{itemize}
    \item How GEM works
    \item Garfild simulation 
\end{itemize}

\subsection{GEM detector}
\subsection{HV}

\begin{itemize}
    \item High Voltage Regestor chain 
    \item HV models 
    \item High Voltage Scan
    \item the lowest voltage that meets the requirements
\end{itemize}

\subsection{LV}

\begin{itemize}
    \item The Low voltage power supply
    \item current consideration 
    \item images of the low voltage
    \item cooling 
\end{itemize}

\subsection{Gas Flow}

    \begin{itemize}
        \item The Gas flow system
        \item bubbler
        \item how to distribute to the chamber 
        \item gas flow in the chamber
    \end{itemize}

\subsection{DAQ system of the GEM detector}

\subsubsection{APV}
\begin{itemize}
    \item APV 25, charge sensitive pre-amplifier
    \item HDMI cable
    \item Noise Reduction technics (the fifth channel of the HDMI is not good)
\end{itemize}

\subsubsection{MPD}

\begin{itemize}
    \item analog to digital converter
    \item CPU 
    \item event rate limitations
\end{itemize}



\subsection{QUATZ / AT}
\subsection{trigger system}

