\chapter{Experiment Setup ~20}

\section{Jefferson Lab CEBAF}

Para. 1
\begin{itemize}
    \item continues beam
    \item linear accelerator
    \item north and south linacs
    \item 12GeV upgrade
    \item Beam energy and current ability
\end{itemize}


Para. 2
\begin{itemize}
    \item north and south linacs each can gen 1.1GeV 
    \item East/West Arc bend the beam to acellerate again 
    \item Hall ABC, 10.9GeV 
    \item Hall D 12GeV
\end{itemize}

Para. 3
\begin{itemize}
    \item PRex experiment choose CEBAF
    \item high current
    \item polarize beam ability
\end{itemize}

Both PRex and CRex experiments are performed on Jefferson Lab Con


\section{Beam injection}

\subsection{Polarized Beam}

\subsection{Helicity Control}

\section{Beam monitor}
\subsection{Polorimeter}
\subsection{Compton}
\subsection{Beam Position Monitor}
\subsection{Beam current monitor}

\section{Target System}


\begin{itemize}
    \item target chamber
    \item target ladder 
    \item different target type and its usage
\end{itemize}

\section{High Resolution Spectrometer}

[intro]

\begin{itemize}
    \item core part of the Hall A 
    \item magnet package
    \item septum magnet
    \item tracking detector (VDC, GEM)
    \item AT
    \item trigger
\end{itemize}

\subsection{Sieve plane}

\begin{itemize}
    \item provides the ground truth dataset for the calibration. (Regression model)
    \item CAD structure of the sieve 
    \item location of the sieve. sieve in / out
    \item merge some of the plot from next chamber
    \item  will give a detailed introduction of how to leverage the sieve to calibrate the spectrometer
\end{itemize}

\subsection{Magnet Chain of the HRS}
\subsubsection{Septum Magnet}
\subsubsection{QQDQ magnet package}
\subsection{Vertical Drift Chamber}

\begin{itemize}
    \item GEM foil
    \item GEM foil structure, microscope image
    \item GEM foil magnetic field simulation (Garfield simulation)
\end{itemize}


\begin{itemize}
    \item GEM detector structure
    \item AI window used de-polarized the window
    \item Window 
    \item 3 GEM layers
    \item Read out strips 
    \item backbone 
    \item back chamber used for balance the pressure, prevent the readout board bend
\end{itemize}


\begin{itemize}
    \item How GEM works
    \item Garfild simulation 
\end{itemize}

\subsection{GEM detector}
\subsection{HV}

\begin{itemize}
    \item High Voltage Regestor chain 
    \item HV models 
    \item High Voltage Scan
    \item the lowest voltage that meets the requirements
\end{itemize}

\subsection{LV}

\begin{itemize}
    \item The Low voltage power supply
    \item current consideration 
    \item images of the low voltage
    \item cooling 
\end{itemize}

\subsection{Gas Flow}

    \begin{itemize}
        \item The Gas flow system
        \item bubbler
        \item how to distribute to the chamber 
        \item gas flow in the chamber
    \end{itemize}

\subsection{DAQ system of the GEM detector}

\subsubsection{APV}
\begin{itemize}
    \item APV 25, charge sensitive pre-amplifier
    \item HDMI cable
    \item Noise Reduction technics (the fifth channel of the HDMI is not good)
\end{itemize}

\subsubsection{MPD}

\begin{itemize}
    \item analog to digital converter
    \item CPU 
    \item event rate limitations
\end{itemize}



\subsection{QUATZ / AT}
\subsection{trigger system}

