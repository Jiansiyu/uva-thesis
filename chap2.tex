\chapter{The Theory of PRex Experiment(~20)}
PRex experiment measures the neutron skin thickness with parity violation asymmetry. This chapter will introduce the theoretical aspect of the experiment. weak interaction, weak interaction form factor, APV, and neutron skin thickness.
% \todo[inline]{Separate the theory and the experiment part into two chapters}
\begin{itemize}
    \item history of the nuclear structure
    \item scatter is used for nuclear structure probe, charge radius 
    \item neutron charges 0, poorly measured
    \item historical measurement 
    \item parity violation measurement of the neutron density
\end{itemize}

The RMS radius of the  neutron distribution in a heavy nucleus  $R_N$ provides an important test of nuclear theory. Furthermore,   $R_N$ is used in the determination of  the density dependence of symmetry energy of neutron-rich matter; this dependence is an  important input in   neutron star structure, heavy iron collision and atomic parity violation experiment calculations. In the past hadron scattering experiments with pion, proton or anti-proton beams have been used to determine the neutron radii of heavy nuclei. However, these measurements suffer from uncertainties associated with the probe particle and the target nucleus. Electron scattering provides a model-independent probe of nuclear radii.  However, in electron scattering, the measurement of neutron distribution in a nucleus  is much harder than the measurement of the proton distribution  since the neutron is uncharged. 

\section{Electromagnetic electron scattering via $\gamma$ exchange}



\section{Proton density}
\subsection{form factor}
\section{electron scattering}
\subsection{EM interaction with the exchange of photon}
\subsection{weak interaction with exchange of $Z^0$}

\section{Parity Violation}
\subsection{Parity}
\subsection{Parity Violation Asymmetry}

\section{Measure neutron density with Parity Violating electron scattering}

% \section{theory aspect}
% \subsection{parity}
% \subsection{weak interaction}
% \subsection{apv}