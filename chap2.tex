\chapter{The Theory of PRex Experiment(~20)}



PRex experiment measures the neutron skin thickness with parity violation asymmetry. This chapter will introduce the theoretical aspect of the experiment. weak interaction, weak interaction form factor, APV, and neutron skin thickness.

\begin{itemize}
    \item history of the nuclear structure
    \item scatter is used for nuclear structure probe, charge radius 
    \item neutron charges 0, poorly measured
    \item historical measurement 
    \item parity violation measurement of the neutron density
\end{itemize}

The RMS radius of the  neutron distribution in a heavy nucleus  $R_N$ provides an important test of nuclear theory. Furthermore,   $R_N$ is used in the determination of  the density dependence of symmetry energy of neutron-rich matter; this dependence is an  important input in   neutron star structure, heavy iron collision, and atomic parity violation experiment calculations. In the past hadron scattering experiments with pion, proton or anti-proton beams have been used to determine the neutron radii of heavy nuclei. However, these measurements suffer from uncertainties associated with the probe particle and the target nucleus. Electron scattering provides a model-independent probe of nuclear radii.  However, in electron scattering, the measurement of neutron distribution in a nucleus  is much harder than the measurement of the proton distribution  since the neutron is uncharged. 


\section{Historical measurement}
\subsection{Pb charged radius measurement and its result}
\subsubsection{calculate with Coulomb energy differences Phys. Lett 29B 396 (1969)}
\subsubsection{L. Ray and G.W.Hoffmann Physics Rev C 31. 538 (1985)}

[ref to Parity Violating Measurements of the Neutron Density, C.J. Horowitz, etc]

\section{Theory background of the PRex experiment}
[Add derive Mott cross section]

The Charge distribution of nuclei is measured by electron scattering. The cross-section can be written as
\begin{equation}
    \frac{d\sigma}{d\Omega} = \frac{d\sigma}{d\Omega_{Mott}}|F_p(Q^2)|^2
\end{equation}

Here is the equation the first part $\frac{d\sigma}{d\Omega_{Mott}}$ is the Mott cross-section with describes the cross-section of electron scattering over a point-like charged particles. The second part is the form factor of the charged particle which contains the structure information of the particle. 


\begin{equation}
    F_n(Q^2) = \frac{1}{4\pi}\int{j_0(qr)\rho_n(r)}d^3r
\end{equation}

In low $Q^2$ approximation, the form factor can be written as:
\begin{equation}
    F(Q^2) = \int{(1-i\Vec{Q}\Vec{r} - \frac{1}{2}(\Vec{Q}\Vec{r})^2)\rho(\Vec{r})}d^3r = Q_e - \frac{1}{6}Q^2<r^2> + ...
\end{equation}




% \section{Electromagnetic electron scattering via $\gamma$ exchange}
% \section{Proton density}
% \subsection{form factor}
% \section{electron scattering} 

\section{Parity Violation Weak interaction}
\subsection{Helicity and Chirality}
\subsection{parity}

In particle and nuclear physics, interactions are described by fundamental forces that are mediated by exchange particles, known as bosons. For a long time, it was believed that all fundamental interactions conserved parity, meaning that the laws of physics remained unchanged under spatial inversion. However, this assumption was challenged by the discovery of parity violation in weak interactions in the 1950s.

[add T.D. Lee and Yang's work]. The groundbreaking experiments by Chien-Shiung Wu and collaborators in 1957 demonstrated that weak interactions, responsible for processes such as beta decay, do not conserve parity. This discovery led to the revised understanding that while electromagnetic and strong interactions conserve parity, weak interactions do not. The recognition of parity violation has had a profound impact on the development of the Standard Model of particle physics and continues to shape research in fundamental physics.

Parity plays a significant role in the classification and understanding of particle and nuclear states. In particular, the determination of parity quantum numbers and their conservation (or violation) in various physical processes has provided important insights into the nature of interactions and symmetries.

 [change needed, add parity plot]

\subsection{weak interaction}

\subsection{weak interaction with exchange of $Z^0$}
\subsection{weak interaction parity violation}
\subsection{electron scattering off Pb}

\subsection{Coulomb distortion}

\section{Measure neutron density with Parity Violating electron scattering}

\section{Target chosen for the experiment}
\section{angle chosen for the experiment}


% \subsection{neutron density theory and corrections to Apv}
% \subsection{nuetron radius measurementys}
% \subsection{neutron stars}
% \subsection{Gravity waves and EOS}


% \section{theory aspect}
% \subsection{parity}
% \subsection{weak interaction}
% \subsection{apv}


