\chapter{The Theory of PRex Experiment(~20)}

\begin{itemize}
    \item history of the nuclear structure
    \item scatter is used for nuclear structure probe, charge radius 
    \item neutron charges 0, poorly measured
    \item historical measurement 
    \item parity violation measurement of the neutron density
\end{itemize}

PRex experiment measures the neutron skin thickness with parity violation asymmetry. This chapter will introduce the theoretical aspect of the experiment: weak interaction, weak interaction form factor, APV, and neutron skin thickness.

The RMS radius of the  neutron distribution in a heavy nucleus  $R_N$ provides an important test of nuclear theory. Furthermore,   $R_N$ is used in the determination of  the density dependence of symmetry energy of neutron-rich matter; this dependence is an  essential input in   neutron star structure, heavy iron collision, and atomic parity violation experiment calculations. In the past hadron scattering experiments with pion, proton or anti-proton beams have been used to determine the neutron radii of heavy nuclei. However, these measurements suffer from uncertainties associated with the probe particle and the target nucleus. Electron scattering provides a model-independent probe of nuclear radii.  However, in electron scattering, the measurement of neutron distribution in a nucleus  is much harder than the measurement of the proton distribution  since the neutron is uncharged. 


\section{Historical measurement[to be added, move to chapter 1]}
\subsection{Pb charged radius measurement and its result }
\subsection{calculate with Coulomb energy differences Phys. Lett 29B 396 (1969)}
\subsection{L. Ray and G.W.Hoffmann Physics Rev C 31. 538 (1985)}
\subsection{ Parity Violating Measurements of the Neutron Density, C.J. Horowitz, etc}

\section{Theory background of the PRex experiment}
[to be optimized]
The charge radius of nucleus can be measured by electron scattering. The cross-section of electrons scattered over charged particles by changing photons can be written as:

\begin{equation}
    \frac{d\sigma}{d\Omega} = \frac{e^4}{4\pi*s}|\frac{M^2}{q^2 - m^2_\gamma + i\epsilon}|^2
\end{equation}

Here in the equation,  e is the electron charge, $s = (k_1 + p_1)^2$ is the Mandelstam variable, $m_\gamma$ is the mass of the photon, and $\epsilon$ is a positive infinitesimal. 

In point-like particle approximation, the potential is the coulomb potential: 
\begin{equation}
    V(r) = \frac{e^2}{r}
\end{equation}

The scattering amplitude can be written as $f(q) = \frac{e^2}{q^2}$, here in the equation $q$ is the four-momentum transfer.

Put the equation in the Klein-Gordon formula, the cross-section can be written as:

\begin{equation}
     \frac{d\sigma}{d\Omega} = \frac{e^4}{4\pi*s}|\frac{M^2}{q^2 - m^2_\gamma + i\epsilon}|^2|f(q)|^2
\end{equation}

In the non-relativistic limit, $s \simeq 4M^2$ and $q^2 = \simeq 4M^2 \sin^2{\theta/2}$, 
\begin{equation}
    \frac{d\sigma}{d\Omega} = \frac{e^4}{4\pi*s} \frac{M^4}{q^4}|f(q)|^2
\end{equation}

\begin{equation}
    f(q) \simeq \frac{e^2}{q^2} - \frac{e^2\alpha}{2q^4}
\end{equation}

Here $\alpha = \frac{e^2}{4\pi}$ is the fine-struture constant:

The cross-section can be written as:
\begin{equation}
      \frac{d\sigma}{d\Omega_{Mott}} =   \frac{e^4}{4\pi*s} \frac{4M^2\sin^4{(\theta/2)}}{(E_1^2 - p_1^2\sin^2{\theta/2})^2}
\end{equation}

Where $E_1$ and $p_1$ are the energy and momentum of the incident electron. 


For no-point-link particles, the Charge distribution of nuclei is measured by electron scattering. The cross-section can be written as
\begin{equation}
    \frac{d\sigma}{d\Omega} = \frac{d\sigma}{d\Omega_{Mott}}|F_p(Q^2)|^2
\end{equation}

Here is the equation the first part $\frac{d\sigma}{d\Omega_{Mott}}$ is the Mott cross-section with describes the cross-section of electron scattering over a point-like charged particles. The second part is the form factor of the charged particle which contains the structure information of the particle. 


\begin{equation}
    F_n(Q^2) = \frac{1}{4\pi}\int{j_0(qr)\rho_n(r)}d^3r
\end{equation}

In low $Q^2$ approximation, the form factor can be written as:
\begin{equation}
    F(Q^2) = \int{(1-i\Vec{Q}\Vec{r} - \frac{1}{2}(\Vec{Q}\Vec{r})^2)\rho(\Vec{r})}d^3r = Q_e - \frac{1}{6}Q^2<r^2> + ...
\end{equation}

Through electron scattering, the charge radius of nuclei has been well measured. [add ref the 208Pb charge radius measurement]

\subsection{concept}
\subsubsection{Helicity}
\subsubsection{Chirality}
\subsubsection{Parity}
In particle and nuclear physics, interactions are described by fundamental forces that are mediated by exchange particles, known as bosons. For a long time, it was believed that all fundamental interactions conserved parity, meaning that the laws of physics remained unchanged under spatial inversion. However, this assumption was challenged by the discovery of parity violation in weak interactions in the 1950s.

[add T.D. Lee and Yang's work]

The groundbreaking experiments by Chien-Shiung Wu and collaborators in 1957 demonstrated that weak interactions, responsible for processes such as beta decay, do not conserve parity. This discovery led to the revised understanding that while electromagnetic and strong interactions conserve parity, weak interactions do not. The recognition of parity violation has had a profound impact on the development of the Standard Model of particle physics and continues to shape research in fundamental physics.

Parity plays a significant role in the classification and understanding of particle and nuclear states. In particular, the determination of parity quantum numbers and their conservation (or violation) in various physical processes has provided important insights into the nature of interactions and symmetries.

 [change needed, add parity plot]
\subsection{Electroweak Interaction}

\section{Parity Violation Weak interaction}

\subsection{weak interaction}
[into to 4 forces, and its mediators]

There are four fundamental forces governing the universe: strong force, weak force, electromagnetic force, and gravitational force. These forces vary in terms of their relative strengths and ranges. Among them, gravity is the most feeble, yet its influence is unbounded, extending infinitely. In contrast, the electromagnetic force, while also limitless in range, is significantly stronger than gravity. The strong and weak nuclear forces operate within a limited range, primarily influencing interactions at the subatomic level. Despite its name, the weak nuclear force is more potent than gravity but pales in comparison to both the electromagnetic force and the strong nuclear force. The strong force, as the name suggests, is the strongest of all four fundamental interactions.

The interaction of three of these fundamental forces—strong nuclear, weak nuclear, and electromagnetic—relies on the exchange of force-carrier particles, classified more broadly as bosons. These particles facilitate the transfer of discrete energy units between matter particles. Each fundamental force has a respective boson: the strong force is transmitted by gluons, photons carry the electromagnetic force, and W and Z bosons mediate the weak force. The corresponding boson for gravity, hypothetically, is the graviton, although its existence remains unconfirmed.

Unlike the photon and gluons, the weak force intermedia W and Z boson are extremely heavy. 

\begin{equation}
    M_w = 80.40 \pm 0.03 GeV/c^2
\end{equation}

\begin{equation}
    M_z = 91.188 \pm 0.002 GeV/c^2
\end{equation}

\begin{table}[!h]
    \centering
    \begin{tabular}{c |c | c}
         Force  & Strength & Mediator \\ \hline
         Strong & 10  & Gluon \\ \hline
         Electromagnetic & $10^{-2}$ &  Photon \\ \hline
         Weak & $10^{-13}$ & W/Z boson \\ \hline
         Gravitational & $10^{-42}$ & Graviton \\ \hline
    \end{tabular}
    \caption{Fundamental Forces}
    \label{tab:my_label}
\end{table}

\subsection{weak interaction with exchange of $Z^0$}
\subsection{ parity violation}
\subsection{electron scattering off Pb}

\subsection{Coulomb distortion}

\section{Measure neutron density with Parity Violating electron scattering}

\section{Target chosen for the experiment}
\section{angle chosen for the experiment}


% \subsection{neutron density theory and corrections to Apv}
% \subsection{nuetron radius measurementys}
% \subsection{neutron stars}
% \subsection{Gravity waves and EOS}


% \section{theory aspect}
% \subsection{parity}
% \subsection{weak interaction}
% \subsection{apv}


