\chapter{Result and Conclusions}

\section{Result of PRex-II experiment}

The PRex-II experiment measures the neutron skin thickness with the parity violated electron scattering. The PRex-II experiment was carried out at Jefferson Lab Hall A with the High-resolution spectrometer. The experiment was performed at a beam energy of 0.95 GeV/c and collected 114 coulombs of charges. 



\begin{itemize}
    \item equations
    \item APV
    \item R_w
    \item symmetry energy (L)
    \item skin thickness 
\end{itemize}


\begin{itemize}
    \item PRex experiment was carried out in jlab 
    \item energy 
    \item  beam current 
    \item  This chapter will introduce about result 
    \item use the skin thickness to get the neutron star, EOS, L 
\end{itemize}

\begin{itemize}
    \item run info
    \item beam info
    \item beam energy 
    \item beam current 
\end{itemize}


\begin{itemize}
    \item measurement errors for each parts
    \item beam polarization 
    \item APV
\end{itemize}

Para. table of different measurement errors (refer)

\begin{table}[]
    \centering
    \begin{tabular}{c c c} \\ 
    \hline
         Correction     & Absolute [ppb] & Relative [$\%$]  \\ \hline
         Beam asymmetry & $-60.4 \pm 3.0$ & $11.0 \pm 0.5$ \\ 
         Charge correction & $20.7 \pm 0.2$ & $3.8 \pm 0.0$ \\
         Beam polarization & $56.8 \pm 5.2$ & $10.3 \pm 1.0$ \\
         Target diamond foils & $0.7 \pm 1.4$ & $0.1 \pm 0.3$ \\
         Spectrometer rescattering & $0.0 \pm 0.1$ & $0.0 \pm 0.0$ \\
         Inelastic contribution & $0.0 \pm 0.1$ & $0.0 \pm 0.0$ \\
         Transverse asymmetry & $0.0 \pm 0.3$ & $0.0 \pm 0.1$ \\
         Detector nonlinearity & $0.0 \pm 2.7$ & $0.0 \pm 0.5$ \\
         Angle determination & $0.0 \pm 3.5$ & $0.0 \pm 0.6$ \\
         Acceptance function & $ 0.0 \pm 2.9$ & $0.0 \pm 0.5$ \\
         Total correction & $17.7 \pm 8.2$ & $3.2 \pm 1.5$ \\
         $A_{PV}^{meas}$ and statistical error & $550 \pm 16$ & $100.0 \pm 2.9$ \\ \hline
    \end{tabular}
    \caption{Correctiuons and systematic uncertainties of PRex [adapted from ...PRex-II]}
    \label{tab:my_label}
\end{table}

[.... todo to be added, density vs radius plot ]

\section{result}

Para.
\begin{itemize}
    \item APV
    \item F_w
    \item R_w
    \item R_n - R_p
    \item $\rho_w$
\end{itemize}


[... plot]

\begin{itemize}
    \item combine the Result from PRex I
\end{itemize}
Para.
\section{impact of the result}
\subsection{EOS, $L$}


\subsection{neutron star radius refer}

\subsection{more ... ?}


[TODO ?? stretched scope]

[to be added... error propagation]
\begin{itemize}
    \item quartz nonlinearity 
    \item beam polarization 
    \item  Transverse 
\end{itemize}



% \begin{itemize}
%     \item PRex result Asym 
%     \item PRex result uncertainty
%     \item neutron radius result
% \end{itemize}

% [....  table of the uncertainty]

% Para. 



% \begin{itemize}
%     \item neutron star  refers to paper
% \end{itemize}

% \begin{itemize}
%     \item EOS measurement from the PRex experiment 
% \end{itemize}

% \subsection{ Parity Violating Measurements of the Neutron Density, C.J. Horowitz, etc}
% \subsection{PRex-I experiment}
% \lipsum[1-20]
